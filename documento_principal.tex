\documentclass[12pt]{article}

\usepackage{sbc-template}

\usepackage{graphicx,url}

%\usepackage[brazil]{babel}   
\usepackage[utf8]{inputenc}  

     
\sloppy

\title{Plano de Governança de TI\\ da Zika Cola}

\author{Guilherme O. de S. Leão, Pedro B. Freitas}


\address{Análise e Desenvolvimento de Sistemas -- Instituto Federal de São Paulo
  (IFSP)
}


\begin{document} 

\maketitle
     
\begin{abstract}
  This document is an IT Governance plan for Zika Cola according to the practices of COBIT, ITIL, PMBOK and CMMI / MPS.BR with a focus on information security.
\end{abstract}     
     
\begin{resumo} 
Este documento é um plano de Governança de TI para a Zika Cola conforme as práticas do COBIT, ITIL, PMBOK e CMMI/MPS.BR com foco na segurança da informação.
\end{resumo}


\section{Introdução}

A ZikaCola é uma empresa de médio porte do setor alimentício, especificamente refrigerantes. Como prestadores de serviços da Zika Cola, decidimos implantar políticas de segurança na empresa, seguindo as práticas do COBIT, ITIL, PMBOK e CMMI/MPS.BR.

O objetivo deste documento é descrever um plano de implantação de segurança da informação na ZikaCola, tendo como foco aplicar as metodologias mencionadas para evitar vazamento de informações em sistemas de CRMs em nuvem, e-mail e nos computadores físicos da empresa. As informações de contratos, campanhas de marketing, margem de lucros e afins são dados sensíveis cujo o vazamento pode gerar problemas de competitividade contra empresas alimentícias de grande porte nacionais e multinacionais.  


\section{Referencial Teórico} 

O modelo de governança de TI descrito nesse documento segue as práticas do COBIT, ITIL, PMBOK e CMMI/MPS.BR, adaptados ao cenário encontrado na empresa Zika Cola. Foram identificados os pontos de maior relevância e com base nisso, foram propostas práticas à realidade apresentada, conforme descrito ao longo do documento.

\section{Desenvolvimento}

Segue a descrição dos problemas enfrentados atualmente na Zika Cola, os pontos de vulnerabilidade e propostas de melhoria.

\subsection{Descrição}

A ZikaCola tem basicamente dois setores, a parte de fábrica, responsável pela produção do refrigerante, e a administrativa, onde se situa a diretoria e setores de marketing e vendas. A empresa armazena a maior parte dos seus dados em um CRM na nuvem e em repositórios também em nuvem associados aos e-mails da organização. Poucos dados são armazenados localmente nos computadores da empresa, geralmente documentos que estão em desenvolvimento.

Essa empresa teve um problema de vazamento de dados sensíveis que estavam no CRM e em repositórios na nuvem, todos os usuários possuem o mesmo acesso aos dados que foram vazados. A perda desses dados foi seguida de perda de alguns contratos para a concorrência, sugerindo que os mesmos foram vendidos.

Junto com a perda dos contratos, houve um rompimento de confiança na empresa, e os funcionários não estão seguros se tudo está sendo ministrado da melhor maneira.

A ideia é melhorar a segurança para evitar futuras perdas financeiras decorrentes desse tipo de vazamento.

\subsection{Análise do Problema}

Nota-se que grande parte do problema se deve ao fato de que todos usuários possuem o mesmo nível de acesso a todas informações da empresa, em muitos casos, nota-se que certos acessos aos sistema são compartilhados entre vários usuários.

Em uma pesquisa interna na empresa, foi constatado que dados da organização frequentemente são armazenados em contas pessoais de colaboradores e que o acesso aos computadores utilizam sempre a mesma senha e usuário.

Foi observado problemas de segurança fixos, a entrada de pessoas no espaço da empresa não é controlado, sendo perfeitamente possível que uma pessoa entre na empresa acesse um computador e consiga acesso a quase todos os dados da empresa. 

\subsection{Possibilidades}

Temos a possibilidade implantar técnicas na empresa para garantir que o sistema de CRM pare de ser falho e para que os funcionários voltem a ter a confiança no trabalho.

\subsection{Proposta de solução}

A equipe de gerentes propôs aplicar praticar de Gestão de Governança para ajudar a empresa Zika Cola a resolver seus problemas de administração e em relação a o CRM, nisso incluindo COBIT, ITIL, PMBOK e CMMI.

\subsection{implantação de Governança de TI}

Na implementação de Governação de TI poderemos utilizar algumas praticas citadas acima, como por exemplo, um dos conceitos do COBIT é dar a responsabilidade para gerentes e gerenciados, fazendo com que assim, os responsáveis por gerenciar as ferramentas não pequem e deixem pequenas brechas para um possível vazamento. Depois, deveria ser escolhido um framework para ser seguido pela empresa, pois é importante se alinhar com um, de acordo com o próprio COBIT. Também nesse âmbito, a empresa deve considerar separar o gerenciamento da governança. Para tudo ser mais organizado.

Outra interação interessante é utilizar de algumas saídas do PMBOK para gerar alguns artefatos que ajudam a organização e gerenciamento da empresa. Para a Zika Cola, um Plano de Gerenciamento de Riscos e um Plano de Gerenciamento de Comunicações. Gerar esses artefatos garante uma precaução e melhor execução dos trabalhos.

\subsection{Proposta de validação}

Para a validação do projeto, a equipe de gerenciamento deve seguir as práticas aplicadas neste documento. COBIT, ITIL, PMBOK e CMMI estando no mercado a muito tempo, as soluções apresentadas já foram estudas por profissionais muito qualificados, então um começo e possível solução é o seguimento dessas ferramentas.

\subsection{Conclusão}

Com a construção deste trabalho fica claro que a aplicação das práticas de Gestão de Governança podem resolver a maior parte dos problemas da Zika Cola. E que aplicando os métodos apresentados além de fazer uma rotina e um caso mais fácil de ser manipulado, a empresa pode fluir de uma maneira muito melhor.

\section{References}

MAGALHÃES, Ivan Luizio; PINHEIRO, Walfrido Brito. Gerenciamento de
serviços de TI na prática: uma abordagem com base na ITIL: inclui
ISO/IEC 20.000 e IT Flex. Novatec Editora, 2007.

SAHIBUDIN, Shamsul; SHARIFI, Mohammad; AYAT, Masarat. Combining
ITIL, COBIT and ISO/IEC 27002 in order to design a comprehensive IT
framework in organizations. In: 2008 Second Asia International
Conference on Modelling and Simulation (AMS). IEEE, 2008. p. 749-753.

ITIL. United Kingdom, 2013. Disponível em: < https://www.axelos.com/bestpracticesolutions/itil/what-is-itil>. What is PRINCE2?. Disponível em: https://www.axelos.com/bestpracticesolutions/prince2/what-is-prince2.

ISACA. COBIT 4.1 Frameworks and Products Brochure. EUA, 2012. Disponível em:
https://www.isaca.org/Knowledge-Center/cobit/Pages/Overview.aspx. 

\end{document}
